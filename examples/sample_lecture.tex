% Sample LaTeX format that the data preparation script can parse
% Adjust the script's environment names to match your actual notes

\documentclass{article}
\usepackage{amsmath, amsthm}

\newtheorem{theorem}{Theorem}
\newtheorem{lemma}[theorem]{Lemma}
\newtheorem{definition}{Definition}
\newtheorem{example}{Example}

\begin{document}

\section{Mathematical Induction}

\begin{definition}[Principle of Mathematical Induction]
Let $P(n)$ be a statement about positive integers. If:
\begin{enumerate}
    \item $P(1)$ is true (base case), and
    \item For all $k \geq 1$, if $P(k)$ is true then $P(k+1)$ is true (inductive step)
\end{enumerate}
Then $P(n)$ is true for all positive integers $n$.
\end{definition}

\begin{theorem}[Sum of First n Natural Numbers]
For all positive integers $n$:
\[ \sum_{i=1}^{n} i = \frac{n(n+1)}{2} \]
\end{theorem}

\begin{proof}
We proceed by induction on $n$.

\textbf{Base case:} When $n = 1$, the left side is $1$ and the right side is $\frac{1 \cdot 2}{2} = 1$. So the formula holds for $n = 1$.

\textbf{Inductive step:} Assume the formula holds for some $k \geq 1$, i.e.,
\[ \sum_{i=1}^{k} i = \frac{k(k+1)}{2} \]

We need to show it holds for $k + 1$:
\begin{align*}
\sum_{i=1}^{k+1} i &= \sum_{i=1}^{k} i + (k+1) \\
&= \frac{k(k+1)}{2} + (k+1) \quad \text{(by inductive hypothesis)} \\
&= \frac{k(k+1) + 2(k+1)}{2} \\
&= \frac{(k+1)(k+2)}{2}
\end{align*}

This is exactly the formula with $n = k + 1$. By the principle of mathematical induction, the formula holds for all positive integers $n$.
\end{proof}

\begin{example}[Verifying the formula]
Let's verify for $n = 5$:
\begin{itemize}
    \item Left side: $1 + 2 + 3 + 4 + 5 = 15$
    \item Right side: $\frac{5 \cdot 6}{2} = 15$ \checkmark
\end{itemize}
\end{example}

\section{Strong Induction}

\begin{definition}[Strong Induction]
Let $P(n)$ be a statement. If:
\begin{enumerate}
    \item $P(1)$ is true, and
    \item For all $k \geq 1$, if $P(1), P(2), \ldots, P(k)$ are all true, then $P(k+1)$ is true
\end{enumerate}
Then $P(n)$ is true for all positive integers $n$.
\end{definition}

\begin{theorem}[Fundamental Theorem of Arithmetic - Existence]
Every integer $n \geq 2$ can be written as a product of primes.
\end{theorem}

\begin{proof}
We use strong induction on $n$.

\textbf{Base case:} $n = 2$ is itself prime, so it is a product of primes (a product with one factor).

\textbf{Inductive step:} Assume every integer from $2$ to $k$ can be written as a product of primes. Consider $k + 1$.

Case 1: If $k + 1$ is prime, then it is a product of primes.

Case 2: If $k + 1$ is composite, then $k + 1 = ab$ for some integers $a, b$ with $2 \leq a, b < k + 1$. By the inductive hypothesis, both $a$ and $b$ can be written as products of primes. Therefore, $k + 1 = ab$ is also a product of primes.

By strong induction, every integer $n \geq 2$ can be written as a product of primes.
\end{proof}

\end{document}
